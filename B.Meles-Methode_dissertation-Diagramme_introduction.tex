\documentclass{standalone}
\usepackage[utf8]{inputenc}
\usepackage[T1]{fontenc}
\usepackage[french]{babel}
\usepackage{lmodern}
\usepackage{tikz}

\usetikzlibrary{shapes, arrows, matrix, calc, fit}

\tikzstyle{startstop} = [rectangle, rounded corners,
minimum width=3cm,
minimum height=1cm,
text centered,
draw=black,
]

\tikzstyle{io} = [trapezium,
trapezium stretches=true,
trapezium left angle=70,
trapezium right angle=110,
minimum width=3cm,
minimum height=1cm, text centered,
text width=3cm,
draw=black,
]

\tikzstyle{reflexion} = [chamfered rectangle,
% cloud callout, aspect=10,
% cloud puffs=21, cloud puff arc=70,
minimum width=3cm,
minimum height=1cm,
text centered,
text width=8cm,
draw=black,
]

\tikzstyle{process} = [rectangle, rounded corners,
minimum width=3cm,
minimum height=1cm,
text centered,
text width=8cm,
draw=black,
]

\tikzstyle{decision} = [diamond,
minimum width=3cm,
minimum height=1cm,
text centered,
draw=black,
]
\tikzstyle{arrow} = [thick,->,>=stealth]

\begin{document}

  \begin{tikzpicture}[node distance=2cm,-latex,
    vhilit/.style={draw=black, thick, dotted, inner sep=1em}]

    \matrix (diagramme) [matrix of nodes, column sep=3em, row sep=5ex] {
      |[process]| {\fbox{§~$1$} \\ \textbf{Définir} le 1\ier{}~terme du
        sujet} &
      \\
      |[process]| {\fbox{§~$n$} \\ \textbf{Définir} le $n$\ieme{}~terme
        du sujet} &
      \\
      |[decision]|
      \begin{tabular}{c}
        Sujet = \\ question ?
      \end{tabular}
      & \\[-1em]
      & |[reflexion]| {Trouver une \textbf{thèse de sens commun} sur la
        notion,
        la relation entre notions, ou la citation} \\
      & |[reflexion]| {Trouver une \textbf{question sur cette thèse}} \\
      |[decision]|
      \begin{tabular}{c}
        Question \\ totale ?
      \end{tabular}
      & \\[4em]
      |[process]| {\fbox{§~$n+1$} \\ Justifier la \textbf{thèse du sens
          commun}} & |[process]| {\fbox{§~$n+1$} \\ Justifier la
        \textbf{thèse
          sous-jacente} au sujet} \\
      |[process]| {\fbox{§~$n+2$} \\ \textbf{Contredire cette thèse}
        à l'aide des § $1$ à $n$} & |[process]| {\fbox{§~$n+2$} \\
        Justifier qu'\textbf{un élément doit être précisé} pour vérifier
        cette
        thèse} \\[4em]
      |[process]| {\fbox{§~$n+3$} \\ \textbf{Question totale}
        correspondant au sujet} & |[process]| {\fbox{§~$n+3$}
        \\
        \textbf{Question partielle} du
        même type que le sujet} \\[4em]
      |[process]| {\fbox{§~$n+4$} \\
        Chaque partie propose une \textbf{réponse \mbox{affirmative} ou
          négative}, avec justification} &
      |[process]| {\fbox{§~$n+4$} \\
        Chaque partie propose un \textbf{candidat \mbox{satisfaisant} la
          thèse sous-jacente}, avec
        justification} \\
    };

    \node (dessous) [below of=diagramme-4-2, anchor=south] {};

    \draw (diagramme-1-1) edge [dotted] (diagramme-2-1);

    \draw (diagramme-2-1) edge (diagramme-3-1);


    \draw (diagramme-3-1) -| node [above,xshift=-6cm] {\textbf{non}}
    (diagramme-4-2);

    \draw (diagramme-3-1) -| node [below,xshift=-6cm] {
      \begin{tabular}{c} (notion unique, \\ groupe de notions, \\
        citation)
      \end{tabular} } (diagramme-4-2);

    \draw (diagramme-5-2) -| (diagramme-6-1);

    \draw (diagramme-3-1) edge node [left,yshift=4em] {\textbf{oui}}
    (diagramme-6-1);

    \draw (diagramme-4-2) edge (diagramme-5-2);



    \draw (diagramme-6-1) edge node [left,yshift=2em] {\textbf{oui}}
    (diagramme-7-1);

    \draw (diagramme-6-1) edge node [right,yshift=2em] {(demande de
      \textbf{décision})} (diagramme-7-1);

    \draw (diagramme-7-1) edge (diagramme-8-1);

    \draw (diagramme-8-1) edge (diagramme-9-1);

    \draw (diagramme-9-1) edge (diagramme-10-1);



    \draw (diagramme-6-1) -| node [above,xshift=-6cm] {\textbf{non}}
    (diagramme-7-2);

    \draw (diagramme-6-1) -| node [below,xshift=-6cm] {(demande de
      \textbf{précision})} (diagramme-7-2);

    \draw (diagramme-7-2) edge (diagramme-8-2);

    \draw (diagramme-8-2) edge (diagramme-9-2);

    \draw (diagramme-9-2) edge (diagramme-10-2);


    \node (definitions) [vhilit, dashed, fit=(diagramme-1-1)
    (diagramme-2-1)] {};

    \node (problematisation) [vhilit, dashed, fit=(diagramme-7-1)
    (diagramme-8-1) (diagramme-7-2) (diagramme-8-2)] {};

    \node (problematique) [vhilit, dashed, fit=(diagramme-9-1)
    (diagramme-9-2)] {};

    \node (annonce) [vhilit, dashed, fit=(diagramme-10-1)
    (diagramme-10-2)] {};



    \node[above,fill=white,yshift=3em] at (diagramme.north)
    {
      \begin{tabular}{c}
        \LARGE \textsc{Structure d'une introduction de dissertation de
        philosophie} \\ {\large (Baptiste Mélès, 19 juin 2025)}
      \end{tabular}
    };

    \node[above,fill=white,yshift=-0.8em] at (definitions.north) {\Large
      Définitions};

    \node[right,fill=white,yshift=-0.8em] at (definitions.east) {
      \begin{tabular}{l}
        Chaque définition doit être \\
        \begin{tabular}{@{\quad}l}
          1. consensuelle ; \\
          2. nécessaire ; \\
          3. suffisante ; \\
          4. non circulaire (\begin{tabular}[t]{@{}l}
            sans synonyme, \\ sans antonyme, \\ sans paronyme),
          \end{tabular} \\
          et justifiée par
          \begin{tabular}[t]{@{}l}
            des exemples, \\ des contre-exemples, \\ des distinctions
            conceptuelles.
          \end{tabular}
        \end{tabular} \\
        \\
      \end{tabular}
    };

    \node[above,fill=white,yshift=-0.8em] at (problematisation.north)
    {\Large Problématisation};

    \node[above,fill=white,yshift=-0.8em] at (problematique.north)
    {\Large Problématique};

    \node[above,fill=white,yshift=-0.8em] at (annonce.north) {\Large
      Annonce du plan};

  \end{tikzpicture}

\end{document}
