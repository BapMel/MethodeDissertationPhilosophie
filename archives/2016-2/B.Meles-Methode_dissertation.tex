\documentclass[a4paper,11pt]{article}

\usepackage[francais]{babel}
\usepackage[utf8]{inputenc}
\usepackage[T1]{fontenc}
\usepackage{lmodern}

\author{Baptiste Mélès}
\title{Méthode de la dissertation philosophique}
\date{Version du 12~janvier 2016}

\newcommand{\cad}{c'est-à-dire}
\newcommand{\apr}{\emph{a~priori}}
\newcommand{\apost}{\emph{a~posteriori}}
\newenvironment{liste}{\begin{itemize}}{\end{itemize}}


% Mes guillemets
\usepackage[babel]{csquotes}
\MakeAutoQuote{«}{»}


\begin{document}
\maketitle

L'objectif de la dissertation de philosophie est de soulever un problème
sur un sujet donné, et d'y proposer une réponse éclairée.


\tableofcontents

\par


\section{Le brouillon}

La composition d'une dissertation a lieu en trois moments~: le
brouillon, la rédation, la relecture. Cette dernière, souvent négligée,
est pourtant cruciale, notamment pour corriger
l'orthographe\footnote{Certains correcteurs sanctionnent explicitement
  d'un ou deux points une orthographe défaillante. Ceux qui ne le font
  pas sont souvent plus sévères~: l'impression générale de négligence
  que délivre la copie les incite à en retirer implicitement bien
  plus.}.

\par

\subsection{Gestion du temps}

Le brouillon est un moment essentiel de la dissertation. Il faut donc
lui consacrer suffisamment de temps, sans pour autant menacer la qualité
de la rédaction.

\par

On dispose généralement de quatre heures en licence pour composer une
dissertation, et de sept heures pour l'agrégation. On doit ménager un
temps important pour la rédaction, car dans la précipitation, il est
presque impossible de réfléchir efficacement. On peut donc consacrer 1h
ou 1h30 au brouillon en licence (donc 2h30 ou 3h pour la rédaction), 3h
pour l'agrégation (donc 4h pour la rédaction).

\par

L'idéal est d'avoir terminé la rédaction avec au moins 15~minutes
d'avance en licence, 30~minutes pour l'agrégation~; on se réserve ainsi
un temps suffisant pour la relecture.

\par

\subsection{Accumulation des idées}

La première chose à faire est de noter sur le brouillon une ou plusieurs
\emph{définitions} pour chacun des termes importants du point de vue du
sens commun.

\par

Ensuite, on cherche dans les définitions quelle est la tension qui
donnera naissance à la \emph{problématique}.

\par

Enfin, il faut noter sur le brouillon toutes les \emph{idées} --- les
thèses, les auteurs, les références --- à mesure qu'elles nous viennent
à l'esprit, sans les sélectionner. Le tri s'effectuera spontanément plus
tard.

\par

\subsection{Composition du plan}

Une fois que l'on a suffisamment d'idées et que leur organisation
commence à se préciser dans notre esprit, on peut passer à la
constitution du plan. 

\par

Chaque partie du plan doit pouvoir être formulée par une thèse
explicite, et, si possible, par des «formules» facilement
reconnaissables (on en trouvera quelques exemples ci-dessous~: la
substance comme substrat, comme fiction, ou comme fonction~; la guerre
comme déchaînement de violence, comme violence rationnelle, ou comme
violence raisonnable~; etc.).

\par

Le plan doit contenir toutes les parties et les sous-parties~; il n'est
pas nécessaire de pousser la subdivision trop loin. Chaque partie ou
sous-partie doit comporter un titre exprimant la thèse locale en cinq à
dix~mots (par exemple «I~- La substance comme substrat», «A) le primat
ontologique de la substance», «B) le primat
chronologique de la substance», «C) le primat
chronologique de la substance», «II~- La substance comme illusion»).

\par

Enfin, dans le plan, on doit noter avec soin la structure de chacune des
transitions. Cette précaution garantit que le passage d'une partie à une
autre ne sera pas artificiel ou simplement rhétorique.

\par

\subsection{Introduction et conclusion}

Une fois le plan terminé, il est recommandé de rédiger intégralement au
brouillon l'introduction et la conclusion. Ainsi, si l'on est pris par
le temps en fin de rédaction, on n'aura plus qu'à recopier la
conclusion, et la dissertation se terminera proprement, même si dans le
développement l'on n'a pas eu le temps d'écrire en détail tout ce que
l'on espérait.




\section{Introduction}

L'introduction doit être la présentation, progressive et détaillée, de
la problématique.

\par

Il vaut mieux éviter d'y citer des noms de philosophes~: ceux-ci sont
rigoureusement étrangers à la problématisation de la question, même si
plus tard ils vous seront évidemment très utiles pour proposer des
réponses. Partir de l'état de la littérature philosophique serait
inverser le juste ordre des choses~: il faut aller des problèmes à la
philosophie, non de la philosophie aux problèmes. Dans l'introduction
--- comme plus tard dans la conclusion --- l'étudiant doit assumer ses
responsabilités, n'engager que soi, mais s'engager totalement. Une bonne
introduction ne contient aucun nom de philosophe.

\par

Une introduction est généralement composée des parties suivantes,
chacune pouvant être présentée en un alinéa~:
\begin{enumerate}
\item l'\emph{amorce} (très facultative)~;
\item l'\emph{analyse} des termes du sujet~;
\item l'exposition d'une \emph{tension} entre les termes du sujet, qui
  mène à la formulation de la \emph{problématique}~;
\item la présentation des \emph{enjeux} de cette problématique
  (facultative)~;
\item l'\emph{annonce du plan}, ou tout au moins de la première partie.
\end{enumerate}

Il faut apporter un soin particulier à l'introduction, et plus tard à la
conclusion, car ce sont les deux parties qui marquent le plus les
correcteurs. Une introduction bancale ou expéditive laissera une
impression négative que le meilleur développement du monde ne saura
dissiper.

\par

Une bonne introduction occupe généralement entre une demi-page (surtout
en licence) et une page entière (principalement pour l'agrégation).
À~plus d'une page et demie, elle commence à trop s'étirer~: les
questions partent dans tous les sens, parce que le candidat n'arrive pas
à resserrer son étude sur une problématique unique.


\subsection{L'amorce}

On préconise parfois de recourir à une amorce avant de définir les
termes du sujet, sous prétexte que l'entrée dans la dissertation est
moins abrupte. On peut ainsi partir d'une anecdote, d'un exemple tiré du
quotidien, d'un exemple historique, etc. Par exemple, pour le sujet «La
guerre», on peut partir d'une comparaison entre deux figures
historiques~:

\begin{quote}
  «Jean Jaurès est mort pour avoir refusé la guerre quand son pays la
  désirait, Jean Cavaillès pour l'avoir acceptée quand son pays y avait
  renoncé~: aujourd'hui ils sont tous deux reconnus comme des
  «justes». De ce constat paradoxal on peut tirer deux interrogations~:
  la première porte sur la nature de la guerre, la seconde sur les
  moyens de son évaluation morale et politique.»
\end{quote}
L'ensemble de la dissertation pourra donc être vu comme la tentative
d'explication de ce simple constat~: que Jaurès et Cavaillès, avec des
comportements apparemment opposés, puissent être l'objet des mêmes
éloges.

\par

Il vaut mieux éviter de partir directement de l'histoire de la
philosophie, en disant par exemple que Hobbes justifie la guerre par
l'état de nature, etc. La dissertation, dans l'introduction, doit pour
ainsi dire s'appuyer sur la fiction que la philosophie n'ait pas
préexisté à notre réflexion. La diversité des opinions philosophiques
n'est jamais un bon point de départ de dissertation~: l'interrogation
sur le sexe des anges a beau avoir suscité bien des opinions contraires,
elle n'en a pas le moindre intérêt pour autant.

\par

Mais l'amorce est hautement facultative. En cas de manque d'inspiration,
il vaut mieux en faire totalement l'économie que de la rédiger
maladroitement.



\subsection{L'analyse des termes du sujet}

\subsubsection{Définition}

Quand on n'utilise pas d'amorce spécifique, l'analyse des termes du
sujet est le début de la dissertation~; dans ce cas, il ne faut pas
hésiter à commencer \emph{ex~abrupto} par la définition des
concepts. L'introduction est alors sobre mais efficace.

\par

Évitez de mentionner explicitement «le sujet» ou «l'intitulé», par
exemple en disant «Ce sujet nous propose de réfléchir sur...» ou «Le
présupposé de ce sujet est...». Commencez directement par l'analyse des
termes.

\par

L'analyse des termes du sujet consiste à prendre chaque terme important
de l'énoncé et à le définir, fût-ce simplement de manière
préalable. Dans le sujet «La guerre», on peut définir en première
approche la guerre comme «le conflit armé entre deux groupes humains».

\par

Mais même en première approche, une définition n'en est pas une si l'on
ne peut aller du concept à la définition, \emph{et surtout} de la
définition au concept\footnote{En termes aristotéliciens, une bonne
  définition doit non seulement énoncer le genre, mais également la
  différence spécifique (\emph{Topiques}, IV, 101b20~; V,
  101b35--102a20)~; c'est cette dernière qui fait souvent défaut.}.
Supposons que l'on dise par exemple «la guerre, c'est le conflit».
Certes, la guerre est un conflit (on peut donc aller du concept à la
définition), mais tout conflit n'est pas une guerre~: il existe
également des conflits entre collègues de travail, entre membres d'une
famille, entre mâles dominants dans un troupeau, et ces conflits ne sont
pas des guerres (on ne peut donc pas aller de la définition au concept).
Il faut donc trouver, parmi l'ensemble des conflits, ce qui distingue la
guerre en particulier. Nous avons retenu deux critères~: le fait que le
conflit oppose des hommes, et qu'il soit armé~; mais d'autres
définitions sont certainement possibles.

\par

Les définitions que vous donnez en introduction doivent être celles du
sens commun. Elles ne doivent surtout pas être celles d'un philosophe et
encore moins présupposer une thèse philosophique particulière. Par
exemple, ne définissez pas «Dieu» comme une entité immanente à la nature
(que vous pensiez ou non à Spinoza) car ce n'est généralement pas en ce
sens que l'on utilise ce terme. Vos définitions en introduction doivent
être œcuméniques et être acceptées comme des évidences par la première
personne rencontrée dans la rue.

\par

Souvent, un terme à définir possède plusieurs significations. Deux cas
de figure se présentent alors. Si toutes les significations sont liées
les unes aux autres, allez du multiple à l'un, c'est-à-dire commencez
par donner les différentes définitions, puis montrez quelle essence
elles ont en commun (par exemple, pour le sujet «La corruption», vous
pouvez chercher une essence commune aux emplois métaphysique et
politique du mot). Si, à l'inverse, les différentes significations sont
relativement indépendantes les unes aux autres, distinguez clairement
les différents emplois et éliminez ceux qui ne sont pas pertinents (par
exemple, pour le sujet «Le corps peut-il être objet d'art~?», vous
pouvez stipuler dès l'introduction que vous entendrez le corps
exclusivement dans le sens de «corps humain» et non dans le sens
métaphysique d'un individu matériel).

\par

Il arrive que tout l'enjeu d'un sujet de dissertation soit précisément
de définir un concept, notamment quand il commence par «qu'est-ce que»~:
«Qu'est-ce que le bonheur~?», «Qu'est-ce qu'agir~?», «Qu'est-ce qu'une
chose~?», etc. Dans ce cas, le concept doit recevoir \emph{deux}
définitions~: une première approximation en introduction, qui représente
ce que l'on entend généralement par ce concept, et une définition
approfondie qui sera donnée en conclusion du devoir. Ainsi, même quand
la définition est l'enjeu même de la dissertation, il faut
impérativement définir le concept dès l'introduction. 

\par

Lorsque le sujet comporte plusieurs concepts («Bonheur et vertu», «Toute
pensée est-elle un calcul~?», «L'histoire est-elle une science~?»,
«Qu'est-ce qu'une action réfléchie~?»), on peut les définir l'un à la
suite de l'autre~:

\begin{quote}
  «Par pensée, on entend généralement l'ensemble de l'activité théorique
  de l'homme. Le calcul, quant à lui, est une démarche déductive
  reposant sur la manipulation de signes.»
\end{quote}

Il faut prendre garde à éviter toute circularité dans la définition. Par
exemple, définir la pensée comme «activité \emph{mentale} du sujet»
serait s'exposer à la question de savoir ce qu'est à son tour
l'«activité mentale»... et à la réponse spontanée~: «l'activité mentale
est l'activité de la \emph{pensée}». La définition est circulaire~! Elle
transformait simplement un substantif («pensée») en adjectif («mental»).
De même, définir l'animal en commençant par dire qu'il est un être
«biologique» ou «doué de vie», «animé» ou «possédant une âme»
(\emph{anima}), ce n'est que déplacer toute la difficulté dans l'un de
ces mots. La définition doit partir du sens commun et être éclairante~;
par exemple, on peut proposer de définir l'animal comme «un être capable
de se déplacer et de viser ses propres fins»~: on a ainsi défini le
concept par des mots strictement plus simples.

\par

Nul n'a mieux résumé que Kant les conditions d'une bonne définition~:
\begin{quote}
  «Les exigences essentielles et universelles requises pour la
  perfection d'une définition en général peuvent être traitées sous les
  quatre moments principaux de la quantité, de la qualité, de la
  relation et de la modalité.»

  \begin{enumerate}
  \item «Selon la \emph{quantité} --- en ce qui concerne la sphère de la
    définition --- la définition et le défini doivent être des concepts
    \emph{réciproques} (\emph{conceptus reciproci}) et par conséquent la
    définition ne doit être ni plus large, ni plus étroite que son défini~;
  \item selon la \emph{qualité}, la définition doit être un concept
    \emph{détaillé} et en même temps \emph{précis}~;
  \item selon la \emph{relation}, elle ne doit pas être
    \emph{tautologique}, \cad{} que les caractères du défini doivent
    être différents de lui-même, puisqu'ils sont les \emph{principes de
      sa connaissance}~;
  \item enfin selon la \emph{modalité}, les caractères doivent être
    \emph{nécessaires} et par conséquent ne pas être du genre de ceux
    que procure l'expérience\footnote{Kant, \emph{Logique}, §107.}.»
  \end{enumerate}
\end{quote}
Le même auteur a même fourni une méthode pour dégager les définitions~: 
\begin{quote}
  «Ces mêmes opérations auxquelles il faut se livrer pour mettre à
  l'épreuve les définitions, il faut également les pratiquer pour
  élaborer celles-ci. --- À~cette fin, on cherche donc 1)~des
  propositions vraies 2)~telles que le prédicat ne présuppose pas le
  concept de la chose 3)~on en rassemblera plusieurs et on les comparera
  au concept de la chose même pour voir celle qui est adéquate 4)~enfin
  on veillera à ce qu'un caractère ne se trouve pas compris dans l'autre
  ou ne lui soit pas subordonné\footnote{Kant, \emph{Logique}, §109.}.»
\end{quote}

% XXX ajouter les conseils d'Aristote, \emph{Topiques}, VI~: lieux pour
% la définition.


\subsubsection{Tension et problématique}

L'analyse des termes du sujet n'est pas un procédé artificiel~: il
possède une réelle utilité dans la construction de la dissertation ---
et en premier lieu, il empêche bien des hors-sujet. C'est en effet de
ces définitions que l'on doit extraire une \emph{tension}, \cad{} un
conflit. Quand le sujet comporte plusieurs concepts, le conflit apparaît
généralement entre eux quand on essaye de les associer~; quand le sujet
comporte un seul concept, le conflit apparaît souvent entre les termes
mêmes de la définition. C'est ce conflit qui engendre la
\emph{problématique}.

\par

Voici un exemple pour le sujet «Toute pensée est-elle un calcul~?»~:
\begin{quote}
  «Par pensée, on entend généralement l'ensemble de l'activité théorique
  de l'homme. Le calcul, quant à lui, est une démarche déductive
  reposant sur la manipulation de signes. Or, l'histoire récente montre
  qu'un nombre croissant d'activités autrefois réservées à
  l'intelligence humaine --- opérations mathématiques, inférences
  logiques, prises de décisions économiques --- se voient déléguées à
  des machines, dont le fonctionnement repose pourtant sur le seul
  calcul. On peut donc s'interroger sur l'existence de limites à cette
  tendance historique. L'activité théorique de l'homme peut-elle être
  simulée tout entière par la simple manipulation de signes qui
  caractérise le calcul~?»
\end{quote}

\par

La problématique ne doit surtout pas être conçue comme elle l'est
généralement, à savoir comme une question qui ressemble vaguement au
sujet que l'on nous a imposé sans toutefois lui être rigoureusement
identique. Voici ce que nous vous recommandons de faire sur votre
brouillon~:

\begin{enumerate}
\item \emph{définition}~: je définis les principaux termes du sujet au
  moyen de concepts strictement plus simples~;
\item \emph{substitution}~: je réécris le sujet en remplaçant chaque
  terme défini par sa définition~;
\item \emph{tension}~: je trouve où réside la tension dans le sujet
  ainsi reformulé et j'en tire la problématique.
\end{enumerate}

\par

Voici un exemple pour le sujet «Dieu a-t-il pu vouloir le mal~?»~:
\begin{enumerate}
\item \emph{définitions} des principaux termes~:
  \begin{itemize}
  \item Dieu~: créateur du monde possédant toutes les perfections~;
  \item le mal~: ce qui ne doit pas être réalisé pour des raisons
    morales~;
  \end{itemize}
\item \emph{substitution} des définitions aux termes définis dans le
  sujet~: un \emph{créateur du monde possédant toutes les perfections}
  a-t-il pu vouloir \emph{ce qui ne doit pas être réalisé pour des
    raisons morales}~?
\item maintenant la \emph{tension} apparaît sans doute plus clairement~:
  comment un être possédant toutes les perfections a-t-il pu vouloir un
  monde apparemment imparfait~?
\end{enumerate}

\par

On peut alors rédiger l'introduction~:
\begin{quote}
  Par Dieu, on entend généralement un être qui d'une part est créateur
  du monde et de l'autre possède toutes les perfections, c'est-à-dire
  toutes les qualités positives à leur degré ultime. Le mal est ce qui
  ne doit pas être réalisé pour des raisons morales. Dieu possédant
  toutes les perfections, il est supposé infiniment bon, et par
  définition ne devrait pas pouvoir accomplir le mal. Un rapide coup
  d'œil autour de nous semble pourtant nous présenter le mal comme l'un
  des principaux ingrédients du monde dont Dieu serait le créateur~:
  partout la guerre, l'injustice, la mort. L'hypothèse de l'existence
  d'un dieu bon est-elle donc compatible avec celle d'un monde
  apparemment mauvais~?
\end{quote}
L'enjeu du devoir sera, dans chacune des parties, de proposer une
réponse à cette question et à elle seule. On peut ainsi proposer dans
une partie l'hypothèse selon laquelle un monde absolument parfait était
irréalisable, dans une autre l'hypothèse selon laquelle notre monde
n'est en réalité pas imparfait comme il semble l'être, etc.

\par

Voici également trois exemples de définitions et de problématiques
différentes pour le sujet «La science»~:

\begin{quote}
  1. (Définition externe, plutôt sociologique)

Une science se présente généralement à nous comme un ensemble
d'assertions qui devrait unanimement être reconnu comme vrai, et que
l'on suppose avoir déjà fait consensus dans une communauté de
spécialistes tels que les mathématiciens, les physiciens ou les
sociologues. Mais le simple consensus ne fait pas la vérité. Existe-t-il
donc à ce présumé consensus (c'est-à-dire de fait) un fondement
nécessaire (c'est-à-dire de droit), qui soit commun à tout ce que nous
appelons couramment des sciences ?


2. (Définition interne, plutôt épistémologique)

Une science est un ensemble de savoirs que l'on peut obtenir, puis
vérifier, selon des principes méthodologiques déterminés à l'avance. Ces
principes sont par exemple les axiomes et les règles de démonstration du
mathématicien~; ou les théories, les concepts et les formules du
physicien~; ou les concepts, les observations et les statistiques du
sociologue. La science n'est donc pas une simple connaissance, c'est une
connaissance par méthode. Ces principes de méthode semblent pourtant
eux-mêmes échapper à tout contrôle, n'étant généralement pas remis en
cause dans le cours normal de la science. À quelles conditions
l'obéissance à des principes de méthode peut-elle donc valoir comme un
garant de vérité ?


3. (Définition naïve et empirique — parfois très efficace)

Nous appelons sciences un ensemble de discours tous tenus pour «vrais»
et pourtant de natures très variées, qui comprend notamment des sciences
pures comme les mathématiques et la logique, des sciences de la nature
comme la physique et la biologie, des sciences humaines comme la
psychologie et la sociologie. Certaines de ces «sciences» semblent
unanimement reconnues comme telles et font autorité, d'autres font
l'objet de débats passionnés — la psychanalyse, l'histoire, le
marxisme~—, tandis que d'autres prétendus savoirs sont presque
unanimement classés parmi les «pseudo-sciences» — l'astrologie,
l'alchimie, la physiognomonie. Existe-t-il donc des critères
universellement valides qui nous permettraient de déterminer avec
certitude si un domaine de savoir relève ou non de la science ?
\end{quote}

Sans tension, il n'est pas de problématique efficace~: sans tension, on
voit difficilement l'intérêt de se poser telle ou telle question --- et
\emph{a~fortiori} d'y répondre. 

\par

La problématique doit être présentée sous la forme d'une question
terminée par un point d'interrogation.  Cette question ne doit pas être
la répétition pure et simple du sujet, si celui-ci était déjà sous forme
interrogative. Par exemple, pour le sujet «Toute pensée est-elle un
calcul~?», la problématique ne doit surtout pas être «Toute pensée
est-elle un calcul~?», mais être reformulée d'une manière éclairée par
les définitions préalables, comme dans l'exemple précédent~: «L'activité
théorique de l'homme peut-elle être simulée tout entière par la simple
manipulation de signes qui caractérise le calcul~?». Entre le sujet et
la problématique, on a progressé~; et ce, grâce aux définitions, qui
permettent de mieux comprendre où se loge véritablement le problème.

\par

Enfin, la problématique doit consister en \emph{une seule} question. On
a parfois la tentation d'en formuler plusieurs~: «L'activité théorique
de l'homme peut-elle être simulée tout entière par la simple
manipulation de signes qui caractérise le calcul~? Les machines
peuvent-elles tout faire~? L'homme sera-t-il remplacé à terme par des
ordinateurs~?». Mais cette succession de questions angoissées témoigne
parfois d'une absence de choix, d'une hésitation entre plusieurs
problématiques, et de leur simple juxtaposition. Le correcteur ne sait
pas si elles sont toutes subordonnées à la première, si elles en
précisent progressivement le sens (et dans ce cas c'est la dernière qui
doit être retenue comme problématique définitive), ou encore si elles
étudient trois aspects d'une seule et même problématique, qui quant à
elle ne serait pas mentionnée. Il faut donc en choisir une seule~; c'est
ce qui garantit l'unité de la dissertation.

\par








\subsection{Annonce du plan}

L'annonce du plan est un sujet sensible entre correcteurs~; mais par
chance, chacun est tolérant avec le parti pris adverse, pourvu qu'il
soit habilement adopté. 

\par

Certains préconisent en effet d'annoncer dès l'introduction le plan
entier, ce qui confère une véritable unité à la dissertation, et montre
que l'étudiant sait dès le début où il va. De plus, cela facilite le
travail du correcteur en lui permettant de s'orienter facilement dans la
copie. 

\par

Mais on peut préférer ne pas «griller toutes ses cartouches» dès la
première page, et ménager un peu de suspens. Il est en effet toujours un
peu étrange d'annoncer la première partie, puis la deuxième, puis la
troisième, puis de revenir à la première pour la développer. À~quoi bon,
si vous avez déjà tout dit~? Mais si vous n'annoncez pas le plan, il
faudra ensuite que les transitions soient irréprochables et
transparentes. Sinon, le correcteur aura du mal à comprendre la
structure de votre copie, et votre note en subira les conséquences.

\par

Dans tous les cas, il faut annoncer au moins la première partie, \cad{}
montrer comment la problématique mène naturellement à envisager un
premier point de vue~:
\begin{quote}
  «Nous verrons dans un premier temps que la diversité et
  l'imprévisibilité de l'activité spirituelle humaine présentent autant
  de résistances à toute réduction de la pensée au calcul.»
\end{quote}

\par

En tout état de cause, il faut éviter à tout prix le lexique du
boucher~: «nous allons traiter cette question en trois parties», ou,
pire, «nous allons examiner trois points de vue». Tout au plus peut-on
annoncer que «notre réflexion connaîtra trois moments successifs»~: on
doit insister sur la continuité de la pensée entre les différentes
parties du plan.


\section{Développement}

Le développement est typiquement constitué de \emph{deux à quatre
  parties}. Avec une seule partie, on reprocherait à l'étudiant de
n'avoir développé d'un point de vue unilatéral~; avec cinq, de n'avoir
pas suffisamment su regrouper ses pensées. Trois parties est certes le
nombre canonique, mais une excellente dissertation peut n'en comporter
que deux, pour peu qu'elle n'ait rien manqué d'essentiel. Rien n'est
pire qu'une troisième partie boiteuse, rajoutée à la hâte pour atteindre
le chiffre magique, et où l'étudiant n'a plus rien d'essentiel à
ajouter.

\par

Chaque partie doit apporter une proposition de réponse à la
problématique. En particulier, il ne faut surtout pas consacrer la
première partie à redéfinir les termes du sujet --- ce qui aurait dû
être fait en introduction --- ou à exposer une thèse qui ne serait que
préalable à la réponse. 

\par

Sur votre brouillon, le titre de chaque partie doit répondre
explicitement à la question qu'est la problématique. Dans la copie, les
premières phrases de chaque partie doivent formuler clairement la thèse
soutenue. Elles peuvent ensuite indiquer brièvement le plan de la
partie, c'est-à-dire annoncer les sous-parties qui la composent.

\par

Chaque partie doit être divisée en \emph{sous-parties}. Ici encore, le
nombre canonique est trois, mais deux ou quatre peuvent tout à fait
convenir si la matière l'exige. Chaque sous-partie doit être un élément
de réponse à la problématique. La première phrase de la sous-partie doit
dire clairement la thèse qui sera soutenue. Ensuite vient
l'argumentation. Enfin, la dernière phrase résume la thèse de la
sous-partie et montre ce qu'elle apporte à l'argumentation de la partie
dans laquelle nous nous trouvons.

\par

On ne saute pas de lignes à l'intérieur d'une partie. On se contente
d'aller à la ligne à chaque nouvelle sous-partie.


\subsection{Types de sujet}

Il existe principalement quatre types de sujet~: 
\begin{enumerate}
\item \emph{un seul concept} (ou une expression)~: «La substance»,
  «L'égalité», «Le génie», «Être impossible», «Voir», «Faire de
  nécessité vertu», etc.
\item \emph{deux concepts} (ou, plus rarement, trois)~: «Substance et
  accident», «Genèse et structure», «Corps et esprit», «Convaincre et
  persuader», «Foi et raison», «Langue et parole», «Conscience et
  inconscient», «Pensée et calcul», «Mathématiques et philosophie», etc.
\item \emph{une question}~: «Toute philosophie est-elle systématique~?»,
  «Peut-on prouver l'existence de Dieu~?», «Peut-on penser l'histoire de
  l'humanité comme l'histoire d'un homme~?», «Ordre, nombre, mesure»,
  etc.
\item \emph{une citation}~: ««Si Dieu existe, alors tout est permis»»,
  ««La science ne pense pas»», ««Pourquoi y a-t-il quelque chose plutôt
  que rien~?»», etc.
\end{enumerate}
Naturellement, différentes formulations peuvent être à peu près
équivalentes~: «Pensée et calcul» et «Toute pensée est-elle un
calcul~?», «Être impossible» et «Qu'est-ce qu'être impossible~?», etc.



\subsubsection{Un seul concept}

Lorsque le sujet porte sur un seul concept, les problématiques les plus
fréquentes sont~:
\begin{enumerate}
\item un problème de \emph{définition}~;
\item un problème d'\emph{existence}~;
\item la discussion d'une \emph{thèse} naturelle sur ce concept.
\end{enumerate}
Par exemple, sur «Être impossible», on peut s'interroger sur la
\emph{définition}, \cad{} sur ce que c'est qu'être impossible~: est-ce
la même chose qu'être contradictoire~?  Et si oui, contradictoire avec
quoi~: les lois logiques, les lois physiques, des lois métaphysiques~?
Sur «La substance», on peut s'interroger sur l'\emph{existence} des
substances en elles-mêmes, et non seulement dans notre pensée. Sur «La
spéculation», on peut discuter la \emph{thèse} assez naturelle et
répandue selon laquelle toute spéculation est nécessairement vaine et
stérile. Mais évidemment, on peut choisir d'autres problématiques pour
chacun de ces sujets~: il n'existe pas une seule bonne problématique par
sujet.



\subsubsection{Deux concepts}

Lorsqu'un sujet comporte deux termes (ou trois, comme «Ordre, nombre,
mesure»), il existe un piège à éviter à tout prix, qui est de traiter le
sujet concept par concept, comme Eltsine mangeait les hamburgers couche
par couche~: par exemple, de traiter, pour «Genèse et structure»,
d'abord la genèse, ensuite la structure, enfin les relations entre
elles. Dans un tel traitement, seule la troisième partie serait dans le
sujet. Il faut traiter d'entrée de jeu les relations entre les deux
notions.

\par

C'est en introduction, et plus précisément lors de l'analyse du sujet,
que l'on étudie chacune des notions pour elle-même~: d'abord la genèse,
ensuite la structure. Mais la problématique doit déjà lier les deux
notions et poser le problème de leur articulation. Ensuite, chacune des
parties du développement doit porter sur la nature de cette relation.

\par

De même, pour traiter le sujet «Mathématiques et philosophie», on ne
séparera pas les analyses sur les mathématiques de celles qui portent
sur la philosophie. Il faut d'emblée étudier, par exemple, si la
philosophie peut adopter une méthode mathématique comme dans
l'\emph{Éthique} de Spinoza, et si certains concepts mathématiques ---
nombre irrationnel, nombre imaginaire, espace à $n$~dimensions etc.~---
peuvent posséder une signification philosophique~; \cad{}, en somme,
quelle est la part de mathématiques dans la philosophie, et quelle est
la part de philosophie dans les mathématiques.



\subsubsection{Une question}

Les sujets qui se présentent sous la forme d'une question sont réputés
les plus faciles, mais il faut bien prendre garde à deux pièges~:
\begin{itemize}
\item que la nécessité de poser la question ait bien été expliquée en
  introduction~: la question ne doit pas paraître arbitraire~;
\item que la problématique ne soit pas la simple paraphrase du sujet.
\end{itemize}


\subsubsection{Une citation}

Lorsque le sujet est une citation, il ne doit jamais être pris au pied
de la lettre. Quitte à jouer sur les mots, les deux sujets suivants
appellent bel et bien des traitements distincts~:
\begin{itemize}
\item «Pourquoi y a-t-il quelque chose plutôt que rien~?»
\item ««Pourquoi y a-t-il quelque chose plutôt que rien~?»»
\end{itemize}
Dans le premier cas, le sujet est une question, tandis que dans le
second il est une citation (de Leibniz). Quand le sujet est une
question, on doit y envisager des réponses (métaphysiques,
scientifiques, phénoménologiques...), et examiner si elles sont
satisfaisantes. Quand le sujet est une citation, on doit se demander ce
qui peut nous amener à poser cette question~; par exemple, quelle est la
spécificité de l'être humain pour qu'il puisse se poser cette question
--- la question contre-factuelle par excellence~?

\par

De même, avec le sujet ««Tous pourris»», il est évidemment hors de
question de développer la thèse selon laquelle tous les hommes
politiques sont corrompus, puis de voir platement que tous les hommes
politiques ne sont peut-être pas corrompus~; mais il faut s'interroger
sur l'existence même de ce slogan, sur les intérêts de ceux qui le
proclament, sur le danger qu'il représente pour la démocratie.

\par

Une citation ne doit donc jamais être prise au pied de la lettre. Elle
doit toujours susciter une interrogation de second degré, sur
l'existence et les conditions de possibilité du discours qu'elle
rapporte.



\subsection{Quelques types de plan}

Il existe un certain nombre de plans récurrents, que l'on peut appeler
plan dialectique, plan de réhabilitation, plan de dégradation, plan
criticiste, etc. Certains d'entre eux seront décrits ci-dessous. Mais il
faut bien se garder de vouloir appliquer un traitement mécanique aux
sujets. Appliqué à toute force à un sujet, un plan inapproprié gâchera
toute la dissertation. Ces quelques plans récurrents sont présentés
seulement à titre de suggestion, mais ce ne sont pas les seuls plans
possibles, et encore moins les meilleurs. Le meilleur plan sera toujours
celui que vous aurez inventé spécifiquement pour tel ou tel sujet.


\subsubsection{Le plan dialectique}

Le plan dialectique est réputé, à tort, le plus philosophique~: à ses
élèves de l'École Normale Supérieure, Louis Althusser proclamait que
tout plan devait représenter d'abord la passion, ensuite la crucifixion,
enfin la résurrection. Le fameux plan par «thèse, antithèse, synthèse»
est effectivement pertinent dans certaines circonstances.

\par

Par exemple, sur le sujet «La substance», on pourrait adopter le plan
dialectique suivant~:
\begin{enumerate}
\item la substance comme \emph{substrat}~: derrière tout phénomène doit
  se trouver une entité permanente, qui soit en même temps le support du
  discours (Aristote)~;
\item la substance comme \emph{fiction}~: on n'a jamais d'expérience de
  la substance, mais seulement de ses manifestations (Berkeley, Hume)~;
\item la substance comme \emph{fonction}~: la substance n'est certes
  jamais connue en elle-même, mais elle doit être pensée pour rendre
  possible une connaissance des phénomènes (Kant).
\end{enumerate}

\par

On a parfois du mal à remplir la première partie d'un plan dialectique.
Comme elle décrit généralement le point de vue du sens commun, il est
difficile d'y trouver de la profondeur. Par exemple, pour un sujet comme
«Le monde extérieur existe-t-il~?», comment peut-on consacrer plus de
deux lignes à dire que, dans la vie de tous les jours, nous considérons
l'existence du monde extérieur comme allant de soi~?

\par

Pour remédier à ce problème, la plus-value que vous apporterez dans la
première partie ne sera pas du contenu, mais de la \emph{structure}. Par
exemple, vous pouvez, dans chacune des trois ou quatre sous-parties de
cette première partie, mettre au jour l'une des raisons que nous avons
de croire à l'existence du monde extérieur~: l'impression de résistance
(le monde ne se comporte pas toujours comme je l'attends ou le désire),
l'existence d'une intersubjectivité (nos rapports avec autrui supposent
un monde commun), l'efficacité pratique de cette croyance... Vous pouvez
ainsi reconstruire le «système implicite» du sens commun, le décrire
comme s'il s'agissait de la pensée d'un philosophe. La structure que
vous aurez ainsi dégagée pourra d'ailleurs vous être très utile en
deuxième partie~: vous pourrez alors démonter, argument par argument,
toutes les bonnes raisons que nous avons de croire à l'existence du
monde extérieur.

\par

Le plan dialectique a pourtant ses inconvénients~:
\begin{enumerate}
\item il est généralement le plan le plus attendu --- or ce qui ne
  surprend pas votre correcteur tend à l'ennuyer, surtout lorsque le
  même plan fade se voit reproduit en trente exemplaires~;
\item le désir de synthèse à tout prix engendre souvent une troisième
  partie extrêmement plate, sans saveur ni force, où l'on s'efforce de
  concilier sans combat la version amollie de thèses
  contradictoires. Souvent la deuxième partie, celle de la critique, est
  celle où l'on a pris le plus de plaisir, et dont la conciliation
  finale est un affaiblissement considérable.
\end{enumerate}

Aussi convient-il parfois de sacrifier le plan dialectique à d'autres
types de plan, présentant plus de vigueur.


\subsubsection{Le plan criticiste}

Le plan criticiste, sous-espèce du plan dialectique, peut convenir pour
des sujets tels que «La substance», «Le moi», «La conscience
collective», «L'universel», «L'histoire a-t-elle un sens~?», etc. ---
typiquement, quand le sujet porte sur une notion transcendante mais
d'usage fréquent. Le plan est le suivant~:
\begin{enumerate}
\item l'\emph{existence} de la chose~;
\item la chose n'est qu'une \emph{illusion}~;
\item on peut faire un \emph{usage régulateur} de la chose, \cad{}
  postuler son existence à des fins théoriques ou pratiques, faire
  «comme si» la chose existait.
\end{enumerate}
Par exemple, voici un traitement classique pour le sujet «La
substance»~:
\begin{enumerate}
\item la substance comme \emph{chose}~: pourquoi et comment nous sommes
  constamment invités à supposer l'existence de substances dans la vie
  quotidienne~;
\item la substance comme \emph{illusion}~: nous n'avons aucune
  connaissance directe de la substance~; celle-ci peut n'être que le
  fruit de notre imagination, une hypothèse métaphysique invérifiable~;
\item la substance comme \emph{fonction}~: cette notion est utile pour
  connaître les phénomènes, et doit être postulée pour permettre le
  progrès de la science. On peut faire «comme si» la substance existait,
  et ainsi mieux connaître le monde.
\end{enumerate}
De même, on peut adopter le plan criticiste pour le sujet «L'histoire
a-t-elle un sens~?»~:
\begin{enumerate}
\item il \emph{existe} un sens de l'histoire~: on constate en observant
  l'histoire un progrès vers l'égalité et la démocratie~;
\item le sens de l'histoire comme \emph{illusion}~: l'histoire est faite
  de contingences, et ce sont les vainqueurs qui réinventent l'histoire
  à leur avantage~;
\item le sens de l'histoire comme \emph{postulat}~: poser l'existence
  d'un sens de l'histoire peut servir de guide à notre action, par
  exemple pour fixer des fins à l'action politique. Cela ne signifie pas
  que l'histoire ait un sens en elle-même, mais si nous décidons d'agir
  «comme si» c'était le cas, alors par nos actes elle acquerra bien un
  sens.
\end{enumerate}
Naturellement, il faut toujours déterminer avec précision à quel intérêt
est soumis le «comme si»~: intérêt théorique (connaître le monde),
pratique (progrès moral), etc.



\subsubsection{Le plan de réhabilitation}

Il arrive qu'un sujet de dissertation corresponde à un concept chargé
d'une forte connotation péjorative~: «L'égoïsme», «L'erreur», «Le
mauvais goût», «L'argument d'autorité», «Les causes finales»,
«L'anachronisme», etc. Un plan dialectique pourrait être ici extrêmement
fade~:
\begin{enumerate}
\item dans une première partie, on \emph{critique} le concept, selon la
  conception commune (l'égoïsme est un intérêt immoral et nuisible à la
  société, l'erreur fait obstacle à la connaissance, le mauvais goût est
  une perversion du goût)~;
\item dans une deuxième partie, on \emph{justifie} ces concepts
  (l'égoïsme est l'intérêt dominant chez l'homme~; l'erreur est parfois
  fertile~; le mauvais goût peut revêtir un intérêt esthétique, par
  exemple dans le kitsch ou chez Warhol)~;
\item dans une troisième partie, on \emph{concilie} avec fadeur les deux
  points de vue précédents (l'égoïsme est parfois bon, mais il ne faut
  pas en abuser~; l'erreur est parfois fertile, mais il faut quand même
  faire attention~; le mauvais goût ne doit quand même pas être
  excessif).
\end{enumerate}

Naturellement, on peut utiliser le plan dialectique de manière plus
fine, y compris avec ces sujets~; mais, mal utilisé, il revient souvent
à ces formes sans force.

\par

Un plan plus puissant est alors le suivant, qui procède à une
\emph{réhabilitation} progressive du concept péjoratif~:

\begin{enumerate}
\item le concept est \emph{nuisible} (l'égoïsme est un intérêt immoral
  et nuisible à la société, l'erreur fait obstacle à la connaissance, le
  mauvais goût est une perversion du goût)~;
\item le concept est \emph{inévitable} (toute action a lieu sur fond
  d'égoïsme, toute connaissance repose sur une erreur, tout goût est
  mauvais)~;
\item le concept est même parfois \emph{bénéfique} ou souhaitable
  (l'égoïsme a des effets profitables, l'erreur fait progresser la
  connaissance, le mauvais goût fait évoluer l'histoire de l'art).
\end{enumerate}

\par

Dans ce dernier plan, il ne s'agit pas d'adopter une thèse conciliant
deux points de vue opposés, mais au contraire d'approfondir
progressivement une thèse forte, selon une véritable montée en
puissance. 

\par

Naturellement, le plan de réhabilitation est difficilement justifiable
dans certains cas~: «L'esclavage», «Le terrorisme», «Le racisme». Ici,
toute idée de réhabilitation serait assez scabreuse.


\subsubsection{Le plan de dégradation}

Symétriquement au précédent, le plan peut consister à dégrader un
concept spontanément perçu comme positif~: «Le désintéressement», «La
sympathie», «La vérité», «La sincérité», «Le bon goût»,
«L'égalité»... On montre alors successivement~:
\begin{enumerate}
\item que le concept est \emph{bénéfique}~;
\item qu'il est \emph{impossible}~;
\item qu'il est même parfois \emph{nuisible}.
\end{enumerate}





\subsubsection{Le plan \emph{ad~hoc}}

Il existe un nombre indéfini de plans \emph{ad~hoc}, parfaitement
adaptés à un sujet, et souvent à un seul, et qui seront bien plus
pertinents que tous les plans génériques --- dialectique,
réhabilitation, dégradation, criticiste --- dont vous aurez entendu
parler. Ce plan est, à chaque fois, à inventer pour la première
fois. S'il demande de l'audace, il est souvent bien plus payant que tous
les autres types de plans.


\subsection{Quelles thèses faut-il soutenir~?}

Les candidats doivent comprendre qu'ils ne sont jamais jugés sur leurs
idées. Le correcteur n'attend pas des copies qu'il lit la confirmation
de ses propres convictions philosophiques. Il veut lire des copies
argumentées. On préfère largement une copie défendant bien une thèse
avec laquelle on n'est pas d'accord à une copie défendant mal une thèse
qui a notre sympathie. N'essayez donc pas de deviner les orientations
philosophiques du correcteur, qui est souvent plus ouvert d'esprit que
vous ne le croyez. Les inspirations kantienne, heideggerienne,
wittgensteinienne, quinienne ne sont ni encouragées, ni bannies~: tout
dépend de la manière dont vous argumenterez vos idées.

\par

Voici deux conseils généraux au sujet des thèses que vous défendrez~:
défendez des thèses non triviales, mais ne cherchez pas l'originalité à
tout prix.

\par

On est souvent conduit, en début de copie notamment, à défendre des
thèses triviales, proches du sens commun~: dire que le mal existe, que
la substance existe, etc. Mais \emph{cela ne doit pas dépasser le début
  de la première partie}. Il faut rapidement passer à des considérations
non triviales. Cela peut se faire notamment de deux manières~: soit en
rompant avec ces apparences, et en montrant que les choses sont en
réalité plus compliquées, que le sens commun est illusoire~; soit en
examinant de manière structurée tous les présupposés de ce point de vue
trivial, en reconstruisant en quelque sorte le «système implicite» du
sens commun. Dans les deux cas, il ne faut \emph{surtout pas s'attarder
  à la surface des choses} (quitte à y revenir plus tard, de manière
justement non triviale)~: c'est ce qui fait toute la différence entre la
dissertation de philosophie et la dissertation de culture générale.

\par

Si vous défendez une thèse non triviale, il vous viendra souvent à
l'esprit, au moment de l'écrire sur la copie, une objection naïve. Dans
ce cas, \emph{écartez-la explicitement}, pour prévenir tout malentendu
et montrer que vous anticipez le sens commun et prétendez montrer
quelque chose de plus ambitieux.

\par

Mais il faut prendre garde également à l'originalité à tout prix. Les
dissertations, surtout en dernière partie, sont parfois le prétexte à
des envolées d'enthousiasme, où le candidat défend des thèses abstraites
dont le principal intérêt est de n'avoir prétendument jamais été
entendues. Le correcteur n'a généralement aucune objection de principe à
cela, à condition que les thèses soient argumentées~: la nouveauté n'a
pas valeur d'argument.

\par

Il faut donc prendre garde à défendre des thèses non triviales, et à les
argumenter. Ce sont les seuls impératifs concernant le contenu de vos
thèses~; sous ces seules réserves, qui sont naturelles, votre liberté
est totale.



\subsection{Comment soutenir une thèse}

Toute thèse doit être \emph{soutenue}, et jamais simplement exposée.


\par

Il n'existe que deux moyens de soutenir une thèse~: soit, \apr{}, en la
fondant sur des principes~; soit, \apost{}, en l'appuyant sur des
exemples. Dans les deux cas, il convient d'éviter toute
\emph{généralisation abusive}.


\subsubsection{Preuves \apr{}~: les arguments}

Supposons que, dans le cadre d'une dissertation sur le thème «Le
désintéressement», on veuille --- provisoirement ou non --- répondre par
que le désintéressement absolu n'existe pas, \cad{} que toutes nos
actions sont fondamentalement intéressées. Une preuve \apr{} pourrait
être la suivante~:
\begin{quote}
  «L'homme est un être vivant~; or, un être vivant ne peut être poussé à
  agir d'une manière déterminée que s'il y est poussé par un intérêt~;
  par conséquent, l'homme est principalement motivé par des intérêts, et
  non par des valeurs morales».
\end{quote}
Matériellement, les prémisses de cet argument sont certes contestables~:
il faut avoir préalablement montré que l'intérêt et la valeur sont
mutuellement exclusifs, et que l'homme est un être vivant exactement au
même titre que les animaux~; mais l'essentiel, de notre point de vue
actuel, réside dans le caractère \apr{} de l'argument. Celui-ci est un
syllogisme formellement valide\footnote{Ce qui, au passage, montre
  l'utilité directe, pour la dissertation, de la logique~: celle-ci
  n'est pas une discipline isolée du cursus, elle est proprement
  philosophique.}.

\par


\subsubsection{Preuves \apost{}~: les exemples}

Les exemples jouent un rôle crucial dans une dissertation. Ils montrent
d'une part que vous possédez une connaissance directe des objets sur
lesquels vous raisonnez, et d'autre part que vous êtes capables de
relier vos thèses philosophiques à des remarques de premier niveau. Dans
une dissertation de philosophie politique, citez des événements
historiques appartenant à des époques variées. Dans une dissertation
d'esthétique, citez des œuvres d'art relevant d'époques et de genres
variés. Dans une dissertation d'épistémologie, donnez des exemples
scientifiques. Dans une dissertation de morale, de philosophie du
langage etc., donnez toujours des exemples concrets. Utiliser des
exemples concrets, c'est montrer que vos thèses se vérifient à même les
choses, et qu'elles ne sont pas séparées du réel qu'elles prétendent
décrire.

\par

La manipulation d'exemples doit donc faire l'objet d'un soin
particulier. Mais il faut pour cela être lucide sur l'apport réel de nos
exemples à l'argumentation~: éviter les généralisations abusives, et
savoir bien user des exemples.

\paragraph{Le danger de la généralisation abusive}

Une preuve \apost{} de la même thèse ne peut être simplement de la forme
suivante~: 
\begin{quote}
  «Un rapide coup d'œil sur l'histoire de l'humanité suffit à nous
  convaincre de la méchanceté originelle de l'homme.»
\end{quote}
La preuve n'est pas convaincante, car de ce qu'il ait existé
\emph{certains} hommes mauvais --- on n'aura effectivement guère de
peine à en trouver --- elle conclut que \emph{tous} les hommes sont
mauvais. En termes logiques, le sophisme repose sur une confusion entre
quantificateurs. La généralisation est abusive.

\par


\paragraph{Le bon usage des exemples}

D'où le problème suivant~: comment peut-on avancer la moindre thèse
\apost{} qui soit en même temps générale, si l'expérience ne nous livre
jamais que du particulier~? Un procédé pourra vous y aider~:
l'\emph{exemple-limite}.

\par

On peut en effet distinguer trois types d'exemples~: l'exemple typique,
l'exemple ordinaire, et l'exemple-limite. L'exemple typique est celui
qui a été choisi avec soin comme illustrant avec une facilité
particulière la thèse que l'on veut défendre. Arguer de Staline pour
affirmer que tous les hommes sont mauvais, c'est se faciliter
outrageusement la tâche~; l'argument n'a strictement aucune valeur. On
peut tout au plus y recourir provisoirement, dans une première approche,
en montrant que l'on utilise consciemment un exemple typique, et surtout
\emph{en l'accompagnant d'une analyse détaillée}~: «l'exemple de Staline
est à cet égard tout à fait caractéristique (ou représentatif), car...».
Pour compenser la facilité de l'exemple, qui joue d'abord en votre
défaveur, l'examinateur attendra que vous en ayez une connaissance
approfondie~; vous montrerez donc bien en quoi la constitution interne
de votre exemple illustre votre thèse de manière non triviale.

\par

L'exemple ordinaire est celui qui puise dans la moyenne des individus
pour montrer la validité de la thèse~: on montrera par exemple comment
l'homme est mauvais au quotidien. La force persuasive est certes plus
grande que pour l'exemple typique, mais non encore absolue, car il peut
exister des personnes exceptionnelles, largement supérieures à l'homme
ordinaire. Comme le précédent, cet argument serait une généralisation
abusive, \cad{} une confusion entre quantificateurs~: «il existe des
hommes intéressés, donc tous les hommes sont intéressés».

\par

Mais montrer que Pierre ou Jean sont mauvais a beaucoup moins de force
que de montrer en quoi Gandhi pouvait être quelqu'un de fondamentalement
intéressé. Parmi les exemples, seul l'exemple-limite, montrant que même
les actions de Gandhi peuvent être justifiées par un intérêt personnel,
a donc une réelle valeur argumentative.  Soutenir une hypothèse par des
exemples n'a de valeur que comme «une vérification de cette hypothèse
sur des cas exemplaires, délibérément choisis comme particulièrement
défavorables à sa démonstration\footnote{Gilles-Gaston Granger,
  \emph{Essai d'une philosophie du style}, Paris, Armand-Colin,
  Philosophies pour l'âge de la science, 1968.}».




\subsubsection{La modalité des thèses}

Un sophisme apparaît régulièrement dans les dissertations~: il consiste
à évoquer la simple possibilité d'une thèse, et, de là, à en conclure la
vérité ou la nécessité. De même que la généralisation abusive était une
confusion entre quantificateurs («il existe des hommes intéressés, donc
tous les hommes sont intéressés»), on peut voir ici une confusion entre
modalisateurs~: «il est possible que tous les hommes soient intéressés,
donc tous les hommes sont intéressés».

\par

Il faut donc prendre garde aux modalisateurs que l'on emploie, et
principalement à ne pas considérer comme avérées des thèses dont on
s'est contenté d'évoquer la possibilité. Assurément, certaines thèses,
notamment dans les philosophies du soupçon comme celle de Nietzsche,
sont condamnées à rester dans le domaine du possible, et sont
difficilement prouvables~: comment prouver en toute généralité que la
nature tout entière est régie par la volonté de puissance~? Nietzsche
lui-même ne le démontre pas, se contentant d'exposer cette
thèse\footnote{\emph{Par-delà bien et mal}, §36.}. Mais parfois la
simple possibilité est suffisante, car elle permet de réfuter la
prétention adverse à la nécessité («le caractère nécessaire de
l'existence d'actions désintéressées est remis en cause par la seule
cohérence de l'hypothèse d'un monde régi par la volonté de
puissance»). Dans tous les cas, une modalité modeste mais légitime a
toujours plus de force qu'une modalité ambitieuse mais usurpée.


\subsection{Comment réfuter une thèse}

Il existe au moins quatre façons de réfuter une thèse~: la première est
\emph{a~posteriori}, les trois autres \emph{a~priori}.

\par

Une première façon de réfuter une thèse est de \emph{produire un
  contre-exemple}. Si quelqu'un soutient la thèse «il n'y a pas d'action
désintéressée», inutile de montrer que \emph{toute} action est
désintéressée~! Il suffit d'exhiber un seul contre-exemple pour la
réfuter complètement.

\par

Une deuxième façon est de montrer une \emph{faille dans le raisonnement
  adverse}. Supposons quelqu'un soutienne la thèse «il n'y a pas
d'action désintéressée» en commettant, comme il arrive souvent, une
erreur de quantificateur («il n'existe pas d'action désintéressée,
puisque nous voyons sans cesse les hommes autour de nous agir selon leur
intérêt») ou une erreur de modalisateur («il n'existe pas d'action
désintéressée, puisqu'il est possible que tout homme ne soit mû que par
son intérêt personnel»). Dans ce cas, montrez explicitement quelle est
la faille, et vous aurez réfuté la démonstration (reste à démontrer la
thèse inverse).

\par

Une troisième façon est d'\emph{attaquer les prémisses} ou les
présupposés du raisonnement adverse. Supposons que quelqu'un nie
l'existence d'actions désintéressées en s'appuyant sur un syllogisme
valide~: «L'homme est un être vivant~; or, un être vivant ne peut être
poussé à agir d'une manière déterminée que s'il y est poussé par un
intérêt~; par conséquent, l'homme est principalement motivé par des
intérêts, et non par des valeurs morales». Vous pouvez réfuter cette
argumentation en rejetant l'une des prémisses -- par exemple en disant
que l'homme ne se réduit précisément pas à son animalité (ou du moins
\emph{pas nécessairement}, ce qui suffit à invalider la conclusion du
syllogisme).

\par

Une quatrième façon est de \emph{critiquer les définitions} des termes.
Si quelqu'un soutient qu'il n'y a pas d'action désintéressée, vous
pouvez critiquer cette thèse en disant qu'elle confond différentes
sortes d'intérêt, qu'il faut en réalité distinguer~: par exemple
l'intérêt personnel, l'intérêt collectif, l'intérêt rationnel...




\subsection{Comment mobiliser l'histoire de la philosophie}

Un philosophe doit toujours être introduit, et savoir s'effacer au bon
moment. Il n'est qu'invité dans votre dissertation~; tout soliste doit
rester aux ordres du chef d'orchestre. En termes concrets, la première
phrase d'un alinéa, où l'on annonce la thèse à venir, et la dernière, où
l'on résume la thèse examinée, doivent être anonymées comme des copies
d'examen, \cad{} ne contenir aucun nom de philosophe. 

\par

Par ailleurs, un philosophe n'est ni un totem, ni un tabou. Une sottise,
même énoncée par Kant, reste une sottise\footnote{Ainsi, dans
  l'\emph{Anthropologie} (II, B), la féminité est définie par deux
  critères~: la conservation de l'espèce (qui implique la crainte et la
  faiblesse), et l'affinement de la culture (qui implique la politesse
  et la tendance au bavardage).}~: un grand nom n'est jamais une
autorité. Aussi toute assertion, même reprise de Kant, doit-elle être
fondée au même titre que si c'était la vôtre.  Une thèse n'est en effet
jamais isolée dans l'œuvre d'un philosophe~: en ceci, elle est toujours
plus qu'une simple citation. Elle s'inscrit dans un système, ou plus
modestement dans un ensemble de raisons, et c'est sur lui qu'il faut la
fonder.

\par

Pour cette raison, une citation, à elle seule, est rarement
éclairante. Elle doit être décortiquée, expliquée, justifiée. Une copie
sans citation, dans laquelle toutes les thèses sont justifiées les unes
par les autres, est largement préférable à un agrégat de citations
supposées transparentes et autosuffisantes. Rien ne saurait donc être
plus nuisible à une dissertation philosophique que le \emph{Dictionnaire
  de citations}, catalogue d'aphorismes certes rhétoriquement habiles,
mais dont la profondeur n'est souvent qu'apparente, et la systématicité
toujours absente.

\par

Un philosophe doit toujours être cité avec la plus grande précision
possible. Il ne suffit pas de dire que Kant a affirmé quelque part
l'existence de connaissances synthétiques \apr{}~: il faut au moins
renvoyer à la \emph{Critique de la raison pure}, voire plus précisément
à son Introduction.

\par

On peut mentionner quelques citations si on a le bonheur de les
connaître par cœur. Mais si l'on a peu de mémoire, un résumé fidèle des
thèses d'un philosophe n'a pas moins de valeur. En outre, les citations
ont souvent un effet pervers~: pour compenser l'effort qu'a nécessité
leur apprentissage, on tend à les mobiliser à tort et à travers.


\subsection{Transitions}

Les transitions ne sont pas une simple exigence rhétorique, mais
obéissent à une véritable nécessité conceptuelle. Elles témoignent en
effet d'une véritable continuité entre les pensées, plutôt que d'une
simple juxtaposition. Une transition procède typiquement en trois
moments~:
\begin{enumerate}
\item \emph{résumer} en une seule phrase la thèse que l'on vient
  d'exposer~;
\item montrer de manière détaillée, et surtout pas de manière symbolique
  ou allusive, ce qui \emph{manque} à cette thèse~;
\item soumettre l'\emph{ébauche} d'une solution, telle qu'elle sera
  développée dans la partie ou la sous-partie suivante.
\end{enumerate}

\par

Chacun de ces trois moments est crucial, mais c'est souvent le second
qui fait défaut~: on change de point de vue sans avoir vraiment montré
pourquoi il était \emph{absolument nécessaire} (et non simplement
possible) de le faire. Si on ne montre pas clairement dans la transition
pourquoi le point de vue adopté jusqu'ici est insatisfaisant et doit
être abandonné, le lecteur n'a strictement aucune raison de lire la
partie suivante.

\par

Où doit-on mettre des transitions~?
\begin{enumerate}
\item à la fin de chaque sous-partie, dans le même alinéa~;
\item à la fin de chaque partie, ce qui mérite souvent un alinéa à
  part~; et ce n'est pas être verbeux que de lui consacrer cinq à dix
  lignes, ou plus.
\end{enumerate}

\par

Par exemple, supposons que nous ayons adopté le plan suivant pour le
sujet «La guerre»~:
\begin{enumerate}
\item la guerre est un \emph{déchaînement de violence}~;
\item la guerre est une violence, mais dirigée par l'intellect~: une
  \emph{violence rationnelle}~;
\item la pertinence de la guerre dépend des valeurs qui la motivent~:
  sous certaines conditions, elle peut devenir une \emph{violence
    raisonnable}.
\end{enumerate}
La transition de la première à la deuxième partie peut être l'alinéa
suivant~:
\begin{quote}
  «Nous avons vu que la guerre pouvait se présenter au premier abord
  comme un déchaînement de violence, s'inscrivant dans la continuité de
  la rivalité entre les individus pour satisfaire leurs besoins naturels
  (boire, manger, respirer...). Mais ce serait méconnaître trois
  distinctions essentielles. D'abord, les belligérants ne sont pas des
  individus, mais des entités plus abstraites et plus larges, à savoir
  des États. Ensuite, les motivations d'une guerre sont rarement
  réductibles aux conditions de la satisfaction des besoins naturels~:
  on entre en guerre pour s'assurer une position économique privilégiée,
  pour acquérir des terres riches en minerais, pour faire coïncider les
  frontiètres politiques de l'«État» avec les frontières culturelles de
  la «nation», pour laver l'humiliation d'une guerre passée, pour
  répandre la liberté révolutionnaire dans le monde entier, pour
  réaliser le communisme international, pour agrandir son «espace
  vital», pour recouvrer la terre de ses ancêtres, etc.~: rien n'animal
  dans toutes ces motivations. Enfin, les moyens d'action sont de plus
  en plus «raffinés»~: loin de la pierre que l'on jette à autrui, on
  fait de plus en plus appel aux dernières avancées scientifiques (armes
  à feu, bombes atomiques, armes chimiques ou bactériologiques). Loin
  d'être un pur et simple déchaînement de violence, la guerre se
  caractérise donc par un appel constant à l'intelligence. Ne faut-il
  pas, dès lors, considérer que la rationalité est aussi essentielle à
  la guerre que la violence~?»
\end{quote}

\par

Lorsque l'on adopte un plan dialectique, l'une des transitions doit être
plus soignée encore que toutes les autres~: celle qui conclut la
deuxième partie et annonce la troisième. Ici, plus de quinze lignes sont
rarement un luxe. Il faut prendre le temps de bien montrer toute la
tension à laquelle on est parvenu, dans sa radicalité. Plus la
contradiction est radicale, plus la résolution est attendue avec
impatience~: il faut savoir susciter l'intérêt du correcteur~!





\section{Conclusion}

\subsection{Une réponse explicite}

Le rôle de la conclusion est simple~: elle doit \emph{répondre à la
  problématique}. Une conclusion ne doit donc pas être simplement un
résumé de la dissertation, mais répondre explicitement à la question
dont elle était partie.

\par

Il faut fuir comme la peste les conclusions sceptiques paresseuses,
comme «on a vu qu'il existait beaucoup de réponses différentes à cette
question» ou «on a vu que cette notion est complexe et comporte de
nombreux aspects». On peut certes conclure sur une impossibilité de
trancher, mais elle doit être argumentée, et non s'appuyer sur la seule
diversité des opinions. La diversité des opinions n'est plus un bon
point d'arrivée de dissertation qu'un bon point de départ.

\par

La conclusion ne doit contenir \emph{aucun nom de philosophe}. C'est
vous qui parlez en votre nom. Ne dites donc jamais~: «en adoptant un
point de vue heideggerien, on peut dire que...». Si vous avez adopté le
point de vue de Heidegger en citant cet auteur à la fin de votre
dernière partie, il est temps maintenant de voler de vos propres ailes~;
vous n'avez plus besoin de Heidegger pour porter les idées que vous vous
êtes appropriées.



\subsection{L'ouvertude du sujet}

Si vous êtes partis d'une amorce, la reprendre en conclusion pour
l'éclairer d'un jour nouveau peut être instructif~; bien manipulé, ce
procédé confère à la dissertation une efficacité qui n'est pas seulement
rhétorique, mais également spéculative~: il montre que vous saviez dès
le départ où vous alliez, et que le cheminement n'a pas été improvisé
ligne après ligne.

\par

Par exemple, sur le sujet «La guerre», on peut faire écho en conclusion
à l'amorce qui comparait Jaurès et Cavaillès~:

\begin{quote}
  «Si le pacifiste Jaurès et le résistant Cavaillès peuvent être tous
  deux considérés comme des justes, c'est que l'opposition formelle de
  la guerre et de la paix n'est pas tenable, sans quoi Jaurès serait
  lâche ou Cavaillès militariste. Il nous faut donc distinguer deux
  sortes de guerres, correspondant à deux sortes de paix. Si Jaurès
  était pacifiste, ce n'était pas par simple refus de la guerre (la paix
  comme absence de guerre, ou \emph{paix négative}), mais au nom d'une
  \emph{paix positive} conçue comme entente entre les peuples. Si
  Cavaillès s'engagea dans la Résistance après l'Armistice, ce n'était
  pas par refus belliciste de l'état de paix, mais au nom d'une paix
  positive --- son avènement dût-il passer par la guerre --- et contre
  la paix négative s'accommodant de l'Occupation et des crimes dont elle
  fut le théâtre. En distinguant ces deux sortes de paix, on peut
  concevoir la proximité de ces deux personnes, qui est d'avoir
  subordonné le problème de la \emph{valeur} de la guerre prise
  absolument à celui de sa \emph{pertinence} dans une situation
  historique précise. Si l'on peut parler de «justes», c'est parce
  qu'ils ne pensèrent pas en opposant simplement guerre et paix, mais
  guerre injuste et paix juste pour Jaurès, guerre juste et paix injuste
  pour Cavaillès.»
\end{quote}

\par

On préconise parfois le recours à l'\emph{ouverture du sujet}. Mais, mal
maîtrisé, le procédé revient trop souvent à aborder soit des problèmes
qui n'ont aucun rapport avec le sujet («car, après tout, qu'est-ce que
la vérité~?...»), soit des problèmes qui auraient dû être traités («une
nouvelle question se pose, qui serait celle des valeurs au nom
desquelles on mène une guerre»). Dans le doute, il vaut mieux éviter ce
procédé, et terminer directement par la réponse à la question~: ici
encore, la sobriété est parfois gage d'efficacité.


\section{Comment les correcteurs lisent les copies}

Savoir sur quels critères vous êtes évalué vous permettra de rédiger des
copies satisfaisant le mieux possibles les attentes du correcteur. 

\par


\subsection{Ordre de lecture}

Voici un exemple de lecture de copie. Le correcteur lit d'abord
l'introduction et la conclusion. À~ce stade, il a souvent une idée de la
note à quatre points près. C'est comme s'il raisonnait par grandes
cases~:
\begin{itemize}
\item une case~A pour les très bonnes copies, de~14 à~20~;
\item une case B pour les copies correctes, de~10 à~14~;
\item une case C pour les copies insatisfaisantes, de~6 à~10~;
\item une case D pour les copies inachevées ou bâclées, en dessous
  de~6.
\end{itemize}

\par

Ayant ainsi provisoirement identifié le profil de la copie, le
correcteur lit le développement, pour voir si les thèses sont
correctement argumentées~: il juge la qualité de la démonstration, la
pertinence des exemples et des références philosophiques. Généralement,
le développement ne fera pas changer la copie de case --- du moins, pas
dans un sens favorable au candidat~: une copie qui commence et qui finit
mal contient rarement un développement éblouissant. Le développement
permet surtout au correcteur de savoir où positionner la copie dans la
case qui lui correspond (A$+$, A$-$, B$+$, B$-$, ...)~; il permet donc
au candidat de gagner jusqu'à quatre~points.

\par


\subsection{Critères d'évaluation}

Voici, dans l'ordre, les questions que le correcteur peut se poser.

\begin{enumerate}
\item Je lis l'introduction.
  \begin{itemize}
  \item Les principaux termes du sujet ont-ils été définis, au moins de
    façon provisoire~?
  \item Le sujet est-il bien problématisé, en partant de la construction
    d'une véritable tension~?
  \item Chacune des parties annoncées répond-elle à la problématique~?
  \end{itemize}
L'introduction permet déjà de savoir si le candidat s'est approprié le
sujet pour le penser de façon personnelle. 

\item Je lis la conclusion.
  \begin{itemize}
  \item La copie est-elle achevée~?
  \item La conclusion répond-elle clairement à la question posée dans
    l'introduction~?
  \item La conclusion est-elle intéressante, c'est-à-dire non triviale~?
  \end{itemize}

\item Je lis le développement.
  \begin{itemize}
  \item La réflexion de chaque partie est-elle structurée en
    sous-parties, dont chacune contient une thèse~?
  \item Chaque thèse est-elle soutenue par une démonstration, ou par un
    exemple suffisamment analysé~?
  \item Le candidat mentionne-t-il les doctrines philosophiques de
    manière détaillée, en évitant l'avalanche de références évoquées de
    manière allusive~?
  \item Les transitions sont-elles pertinentes~?
  \end{itemize}
\end{enumerate}


\section{Sujets de dissertation}

Voici des sujets pour s'entraîner à la dissertation. Pour chacun d'eux,
rédigez~:
\begin{enumerate}
\item une introduction~: définitions, tension, problématique~;
\item un plan détaillé (aucun nom de philosophe ne doit apparaître dans
  les titres des parties et sous-parties)~;
\item une courte conclusion répondant clairement à la problématique.
\end{enumerate}

\subsection{L'art}

\begin{liste}
  \item Pourquoi conserver les œuvres d'art~?
  \item L'art imite-t-il la nature~?
  \item L'éducation esthétique
  \item L'inspiration
  \item L'artiste sait-il ce qu'il fait~?
  \item L'art et la morale
  \item Le plaisir esthétique suppose-t-il une culture~?
  \item La virtuosité
  \item Qu'est-ce qu'une œuvre ratée~?
  \item Y a-t-il un progrès en art~?
  \item Le génie
  \item Le mauvais goût
  \item Arts de l'espace et arts du temps
  \item L'art engagé
  \item La pluralité des arts
  \item La vérité de l'œuvre d'art
\end{liste}

\subsection{Logique et épistémologie}

\begin{liste}
\item Mécanisme et finalité
\item Le symbolisme mathématique
\item Le hasard n'est-il que la mesure de notre ignorance~?
\item Comment choisir entre plusieurs hypothèses~?
\item La logique nous apprend-elle quelque chose sur le langage
  ordinaire~?
\item La causalité
\item Sauver les phénomènes
\item Les genres naturels
\item Qu'est-ce qu'un nombre~?
\item La cohérence est-elle un critère de vérité~?
\item Des événements aléatoires peuvent-ils obéir à des lois~?
\item L'intuition en mathématiques
\item La contradiction
\item La logique a-t-elle une histoire~?
\item Y a-t-il plusieurs logiques~?
\item La méthode
\item Savoir et pouvoir
\end{liste}


\subsection{La métaphysique}

\begin{liste}
\item L'impossible
\item L'être et le temps
\item Y a-t-il une connaissance métaphysique~?
\item Seul le présent existe-t-il~?
\item N'y a-t-il qu'un seul monde~?
\item L'existence se démontre-t-elle~?
\item Avons-nous une âme~?
\item L'infinité du monde
\item Que prouvent les preuves de l'existence de Dieu~?
\item Le virtuel et le réel
\item Le réel est-il rationnel~?
\item Être et être pensé
\item Penser sans corps
\item Le miracle
\item Logique et métaphysique
\item Dieu a-t-il pu vouloir le mal~?
\end{liste}


\subsection{La morale}

\begin{liste}
\item Sommes-nous responsables de notre passé~?
\item Le repentir
\item Peut-on conclure de l'être au devoir-être~?
\item L'intolérable
\item Le péché
\item La beauté morale
\item Peut-on vouloir le mal~?
\item La morale peut-elle être fondée sur la science~?
\item Y a-t-il un devoir d'être heureux~?
\item La morale peut-elle se passer d'un fondement religieux~?
\item La moralité n'est-elle que dressage~?
\item La morale peut-elle être un calcul~?
\item Le moi est-il haïssable~?
\end{liste}


\subsection{La politique}

\begin{liste}
\item Guerre et politique
\item La rationalité politique
\item Qu'est-ce qu'un contre-pouvoir~?
\item Le totalitarisme
\item Que faut-il savoir pour gouverner~?
\item Le législateur
\item Le respect des institutions
\item Les droits de l'homme sont-ils une abstraction~?
\item La meilleure constitution
\item A-t-on des droits contre l'État~?
\item Qu'est-ce qu'un programme politique~?
\item Y a-t-il des erreurs en politique~?
\end{liste}


\subsection{Les sciences humaines}

\begin{liste}
\item Histoire et ethnologie
\item Les sciences humaines permettent-elles de comprendre la vie d'un
  homme~?
\item Les sciences humaines sont-elles dangereuses~?
\item Expliquer et comprendre
\item Qu'est-ce qui rend l'objectivité difficile dans les sciences
  humaines~?
\item sciences humaines et philosophie
\item L'efficacité thérapeutique de la psychanalyse
\item La psychanalyse est-elle une science~?
\item Sciences humaines et liberté sont-elles compatibles~?
\item Y a-t-il une causalité historique~?
\item L'objectivité de l'historien
\item L'arbitraire du signe
\item Machines et mémoire
\item Les sciences humaines permettent-elles d'affiner la notion de
  responsabilité~?
\item L'économie a-t-elle des lois~?
\item L'argent
\item Y a-t-il un inconscient collectif~? 
\end{liste}


\end{document}
